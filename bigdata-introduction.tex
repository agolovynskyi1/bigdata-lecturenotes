\section{Вступ}

\begin{ozn}
 Великі дані (Big Data) це неструктуровані, слабкоструктуровані та структуровані дані, з якими не можна працювати стандартними математичними методами. Слово ``великі'' означає, що їх обробка також потребує спеціальних методів паралельного програмування та відповідно багатопроцесоних обчислювальних систем.
\end{ozn}

Це маркетинговий термін, який дозволяє розвивати і продавати математичні методи бізнесу. 

Прикладами великих даних є архіви інтернет-форумів, записи даних відеоспостереження торгового центру, дані з чеків покупок да ідентифікатори дисконтних карток, літературні тексти, звукові файли тощо.

Не відносяться до великих даних, незалежно від розміру, добре структуровані дані з чіткою і відомою моделлю фізичного процесу, в яких можна напряму застосувати зрілу математичну теорію, з широкого асортименту розроблених на останні 300 років.

Основний фокус досліждень полягає у роботі з неструктурованими даними з низькою щільністю корисної інформації та отримання з них структурованих даних, це називається Data Mining, дослівно ``видобування даних''.

Задачі, які вирішуються у даній області -- це перевірка, тобто підтвердження чи спростування певної гіпотези. Наприклад, 
\begin{itemize}
 \item чи є певна людина потенційним покупцем?
 \item чи залежать дані процеси від заданого параметра?
 \item яка частина даних несе інформацію про дане явище чи процес?
\end{itemize}

Гіпотеза дає апріорну оцінку, що є потенційно корисною інформацією, а що ігноруються.

Наприклад, деяка компанія виробляє продукцію і продає її по всьому світу. І хоче оцінити симпатії, оцінки споживачів у різних країнах. 

Прямим методом буде визначення цільової аудиторії, проведення опитування у даних країнах, аналіз отриманих анкет.

Непрямим методом буде аназіл постів у соцмережах, побудова векторних просторів (для кожної країни і мови) із спеціальною метрикою, і визначення близькості назви продукту до слів-маркерів. Це ї є великі дані. Відмітимо, що для вирішення даної задачі навіть не потрібно знати мову цих країн.

