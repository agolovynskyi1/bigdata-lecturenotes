\section{Типи даних за структурою}

\begin{ozn}
Структуровані дані: дані, що зберігаються у рядках та стовпцях, здебільшого числові, де чітко визначено значення кожного елемента даних. 
\end{ozn}

Цей тип даних становить близько 10\% від загального обсягу даних на сьогодні. Такі дані доступні через системи управління базами даних. Прикладні джерела структурованих (або традиційних) даних включають 

\begin{itemize}
 \item офіційні реєстри, які створюються урядовими установами для зберігання даних про осіб, підприємств та нерухомість;
 \item дані датчиків у промисловості, які збирають інформацію на заводах про процеси, для контролю руху, температури, розташування, світла, вібрації, тиску, рідини та потоку.
\end{itemize}

\begin{ozn}
Неструктуровані дані: дані різних форм, наприклад, текст, зображення, відео, документи тощо. 
\end{ozn}

Це можуть бути скарги клієнтів, контракти або внутрішні електронні листи. Цей тип даних становить близько 90\% даних, створених у 21 столітті. Вибухове зростання соціальних медіа (наприклад, Facebook та Twitter), починаючи з середини минулого десятиліття, є причиною більшої частини неструктурованих даних, які ми маємо сьогодні. Неструктуровані дані не можуть оброблятися за допомогою методів традиційних реляційних (табличних) баз даних. Важливість неструктурованих даних полягає у схованих у них зв'язках, які можуть бути не виявлені, якщо розлядати лише структуровані дані (наприклад, інформація про людину в соціальній мережі набагато повніша, ніж анетна інформація, яка подаєтья на візу чи у відділ кадрів). Саме це робить дані, створені в соціальних медіа, відмінними від інших типів даних, це те, що дані в соціальних медіа мають сильний відбиток особистості.

\begin{ozn}
Географічні дані: дані, пов'язані з дорогами, будівлями, озерами, адресами, людьми, робочими місцями та транспортними маршрутами, які генеруються з географічних інформаційних систем. 
\end{ozn}

Ці дані пов'язують місце, час та атрибути (тобто описову інформацію). Географічні дані, які є цифровими, мають величезні переваги перед традиційними джерелами даних, такими як карти, письмові звіти дослідників та розмовні записи, в яких цифрові дані легко копіювати, зберігати та передавати. Що ще важливіше, їх легко трансформувати, обробляти та аналізувати. Такі дані корисні для містобудування та моніторингу впливу на навколишнє середовище. Гілка статистики, яка бере участь у просторових або просторово-часових даних, називається Геостатистика.

\begin{ozn}
Мультимедійні дані: потокове передавання в реальному часі живих або збережених медіа-даних. 
\end{ozn}

Особливою характеристикою засобів масової інформації в режимі реального часу є величезні об'єми відео, зображень та аудіо, які в майбутньому будуть лише зростати. Одним з основних джерел медіа-даних є такі сервіси, як, наприклад, YouTube, відеоконференції.

\begin{ozn}
Тести природньою мовою: дані, що генеруються людьми у текстовій формі. 
\end{ozn}

Такі дані різняться за рівнем абстракції та рівнем редакційної якості. До джерел даних природної мови відносяться пристрої збору мовлення, наземні телефони, мобільні телефони та Інтернет речей, які створюють великі розміри текстового зв'язку між пристроями.

\begin{ozn}
Часовий ряд: послідовність точок даних (або спостережень), як правило, що складається з послідовних вимірювань, проведених за часовий інтервал. 
\end{ozn}

Мета - виявити тенденції та аномалії, визначити контекст та зовнішні впливи та порівняти індивіда з групою або порівняти окремих людей у ​​різний час. Існує два види даних часових рядів: 

\begin{itemize}
 \item безперервне, де ми спостерігаємо в кожен момент часу, 
 \item де ми спостерігаємо через (зазвичай регулярно) проміжки інтервалів. 
\end{itemize}

Прикладами таких даних є океанські припливи, кількість сонячних плям, вартість валют, цінних паперів на біржі, та вимірювання рівня безробіття кожного місяця року.

\begin{ozn}
Дані про події: дані, згенеровані в результаті зіставлення зовнішніх подій із часовим рядом.  
\end{ozn}

Це вимагає відокремлення важливих подій від неважливих. Наприклад, інформацію, пов'язану з аваріями транспортних засобів або аваріями, можна збирати та аналізувати, щоб зрозуміти фактори, які спричинили подію і її наслідки. Дані в цьому прикладі формуються датчиками, закріпленими в різних місцях кузова транспортного засобу. Дані про подію складаються з трьох основних фрагментів інформації: 

\begin{itemize}
 \item дія, яка є самою подією, 
 \item часова мітка, час, коли ця подія сталася,
 \item стан, який описує всю іншу інформацію, що стосується цієї події.
\end{itemize}

Дані про події зазвичай мають різноманітну структуру, неунормовані значення, вкладену структуру.

\begin{ozn}
Дані про мережу: дані стосуються структури зв'язів дуже великих мереж, таких як соціальні мережі (наприклад, Facebook і Twitter), інформаційні мережі (наприклад, всесвітня павутина), біологічні мережі (наприклад, біохімічні, екологічні та нейронні мережі) та технологічні мережі (наприклад, Інтернет). Такі дані представлені графом із вузлами, з'єднаними зв'язками різних типів. Об'єктом дослідження стає саме структура мережі та характер зв'язків.
\end{ozn}

\section{Типи даних за значеннями}

Якщо розглянути величини елементів даних, то вони поділяються на категорні (або іменні), порядкові та числові. Нижче ми визначимо ці терміни та пояснимо, чому вони важливі.

\begin{ozn}
Категорна величина (іноді її називають іменною) - це така, яка має дві або більше категорій, але немає внутрішнього упорядкування категорій. 
\end{ozn}

Наприклад, стать - це категоріальна змінна, що має дві категорії (чоловіча та жіноча), і не має внутрішнього впорядкування між ними. Колір волосся - це також категорна величина, що має ряд категорій (блондинка, каштанова, брюнетка, руда тощо), і знову ж таки, немає змістовного способу їх впорядкування від найвищого до нижчого.


\begin{ozn}
Порядкова величина схожа на категорну величину. Різниця між ними полягає в тому, що є чітке впорядкування змінних. 
\end{ozn}

Наприклад, припустимо, що ви маєте змінний економічний статус з трьома категоріями (низький, середній та високий). Окрім того, що ви можете класифікувати людей на ці три категорії, ви можете впорядкувати категорії як низькі, середні та високі. Тепер розглянемо таку величину, як навчальний досвід (із такими значеннями, як випускник початкової школи, випускник середньої школи, якийсь випускник коледжу та коледж). Їх також можна замовити як початкову школу, середню школу, якийсь коледж, так і випускник коледжу. Незважаючи на те, що ми можемо упорядкувати їх від найнижчого до найвищого, інтервал між значеннями може бути не однаковим для рівнів величин. 

Скажімо, ми присвоюємо бали 1, 2, 3 і 4 цим чотирьом рівням освітнього досвіду, і ми порівнюємо різницю в освіті між категоріями першою і другою з різницею навчального досвіду між категоріями два і три, або різницю між категоріями три і чотири. Різниця між категоріями першої та другої (початкова та середня школа), ймовірно, набагато більша, ніж різниця між категоріями дві та три (середня школа та деякі коледжі). У цьому прикладі ми можемо упорядкувати людей за рівнем освітнього досвіду, але розмір різниці між категоріями невідповідний (оскільки інтервал між категоріями одна і дві більший, ніж категорії дві та три). Якби ці категорії були однаково розташовані, то змінна була б числовою змінною.

\begin{ozn}
Числова величина схожа на порядкову змінну, за винятком того, що інтервали між значеннями числової змінної однаково розташовані. 
\end{ozn}

Наприклад, припустимо, у вас є така величина, як річний дохід, яка вимірюється в доларах, а у нас є троє людей, які заробляють 10 000, 15 000 і 20 000 доларів. Друга людина заробляє на 5000 доларів більше, ніж перша особа, і 5000 доларів менше, ніж третя особа, і розмір цих інтервалів однаковий. Якби були ще двоє людей, які б заробляли 90 000 доларів і 95 000 доларів, розмір цього інтервалу між цими двома людьми також був би однаковий (5000 доларів).

Одна і та сама величина може бути різною за типом, відповідно до контексту. Наприклад, число може відноситить як до числової величини, так і до категорної, якщо це номер комунікаційного порту, ідентифікатор транспортного засобу тощо.

\section{Вектори і тензори}

\begin{ozn}
Скаляр (від лат. scalaris — східчастий) -- величина, кожне значення якої може бути виражене одним числом.
\end{ozn}

\begin{ozn}
Вектор (від лат. vector, «той що несе») -- (послідовність, кортеж) однорідних елементів.
\end{ozn}

\begin{ozn}
Матриця -- сукупність математичних величин, певним способом розміщених у прямокутній таблиці.
\end{ozn}

\begin{ozn}
Тензор (від лат. tendere, «тягнутись, простиратися») -- тензор представляється у вигляді багатовимірної таблиці, заповненої числами.
 \end{ozn}

Для об'єктів однакової розмірності природньо визначаються операції додавання і множення на скаляр.

\section{Кодування}

\subsection{Унітарний код}

Нехай є набір $X$ з $n$ елементів. 

Унітарний код (англ. one-hot encoding) -- це таке відображення, коли кожному елементу $x \in X$ ставиться у відповідність вектор $e \in R^n$,  що містить тільки одну 1, яка відповідає позиції $x$ наборі.

\subsection{Кодування зображень}

Для кодування кольорових графічних зображень застосовується принцип декомпозиції кольору на основні складові: червоний (Red), зелений (Green) і синій (Blue). Цей принцип базується на тому, що будь-який колір можна отримати шляхом змішування трьох зазначених кольорів. Система кодування за першими літерами назв основних змішуються квітів називається системою RGB і описує поведінку адитивної моделі кольорів.  Якщо для кодування яскравості кожної складової кольору використовувати 256 градацій (8-розрядне число), то для кодування кольорової точки досить 24-розрядного двійкового числа.

Таким чином, зображення може бути представлене тензором $H \cdot W \cdot R \cdot G \cdot B$. Кожна величина у тензорі має числовий тип.

Lab -- система задання кольорів, що використовує як параметри світлосилу, відношення зеленого до червоного та відношення синього до жовтого. Ці три параметри утворюють тривимірний простір, точки якого відповідають певним кольорам. 

Приклад. Представлення зображень поняття ``яблуко'' у тензори.

Афінський філософ Платон (427-347 рр. до н.е.). Основну  частину  всієї  філософії  Платона  займає теорія ідей.  Філософ  визначає  наявність існування  двох світів -- ідей та речей.  Ідеї  (від  грец. ейдос)  є  ніщо інше  як  прообрази  або  ж  інакше  кажучи -- витоки  речей.  У свою  чергу   Платоном  висувається  думка стосовно  того,  що  в основі  безлічі  речей,  які  утворені   від  безформної  матерії  покладені  ідеї.  Вони  виступають   джерелом  усього, а   матерія  здатності  породження  немає.
