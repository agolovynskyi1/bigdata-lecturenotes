\section{Часові ряди}

\subsection{Кореляція і автокореляція}

Нехай є дві вибірки скалярних величин $X \in \R , Y\in \R, x_i \in X, y_i \in Y, i \in 1, 2, \ldots, n$, із середніми відповідно $\avg{x}, \avg{y}$ і стандартними відхиленнями $\sigma_x, \sigma_y$.

\begin{ozn}
Коваріацією виборок $X, Y$ буде величина 
\begin{equation}
 cov(X,Y) = M((X - \avg{X})(Y - \avg{Y}) = \frac{1}{n} \sum_{i = 1}^{n} (x_i - \avg{x})(y_i - \avg{y})
\end{equation}
це міра спільної мінливості двої випадкових величин.

Використовуючи властивість лінійності математичного сподівання

\begin{equation}
\begin{split}
 cov(X,Y) = M((X - M(X))(Y - M(Y)) = \\
 = M(XY - XM(Y) - M(X)Y + M(X)M(Y)) = \\
 = M(XY) - M(X)M(Y) - M(X)M(Y) + M(X)M(Y) = \\
 = M(XY) - M(X)M(Y) = \frac{1}{n} \sum_{i = 1}^{n} x_i y_i - \avg{x} \avg{y}
\end{split}
\end{equation}

Якщо X та Y є незалежними, то їхня коваріація є нульовою. Це випливає з того, що за незалежності

\begin{equation}
M(XY) = M(X)M(Y)
\end{equation}



Якщо $X\in \R^k, Y\in \R^m$ випадкові величини у векторній формі, аналогічно визначимо взаємно-коваріаційну матрицю
\begin{equation}
cov(X,Y) = M((X - M(X))(Y - M(Y)^T) = M(XY^T) - M(X)M(Y)^T
\end{equation}

$(i, j)$-тий елемент цієї матриці дорівнює коваріації $cov(X_i, Y_j)$ між i-тою скалярною складовою X та j-тою скалярною складовою Y.
\end{ozn}


\begin{ozn}
Кореляцією виборок $X \in \R, Y \in \R$ буде величина
\begin{equation}
corr(X,Y) = \frac{cov(X,Y)}{\sigma_x \sigma_y} = 
\end{equation}
це коваріація, унормована на стандартні відхилення.
\end{ozn}

Нехай є випадкова величина $X \in \R$, для спрощення формул нехай $\avg{X} = 0, \sigma_X = 1$, тобто дані є центрованими і нормованими. 

Визначимо вектор $X_i = (x_i, x_{i+2}, \ldots, x_{i+n}) $ з $n$ елементів, $ x_i \in X$ і зміщення $\tau \in \N$.

\begin{ozn}
Автоковаріацією називається величина 
\begin{equation}
cov(X_i, X_{i+\tau}) = M(X_iX_{i+\tau}^T) = \frac{1}{n} \sum_{i = 1}^{n} x_i x_{i+\tau}
\end{equation}
\end{ozn}

Пошук максимума функції автоваріації за $\tau$ дозволяє знайти точні позиції шаблону у послідовності даних. 

Якщо є гіпотеза, що випадкова величина має періодичний характер, але не відомими є величини періодів,
при дослідженні залежності величини автоковаріації від ширини вікна $n$ і $\tau = n$, максимуми будуть відповідати прихованим періодам.


Задача 1. Пошук прихованої перідодичності у даних, представлених у вигляді часових рядів. Дані брати пов'язані із людською активністю.

